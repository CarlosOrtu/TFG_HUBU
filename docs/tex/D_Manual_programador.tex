\apendice{Documentación técnica de programación}

\section{Introducción}

En este anexo se va explicar la estructura del proyecto, los requisitos necesarios para poder ejecutarlo desde cualquier ordenador  

\section{Estructura de directorios}

El contenido del repositorio se estructura principalmente en 4 directorios:

\begin{itemize}
    \item \textbf{/Base de datos sintética:} Contiene el script para generar la base de datos sintética.
    \item \textbf{/Lavarel:} Contiene el proyecto de la aplicación web junto con los test, las migraciones, los seeders y el enrutado.
    \item \textbf{/SQL:} Contiene los scripts SQL que crean la estructura de la base de datos e introducen los datos necesarios para que funcione la aplicación.
    \item \textbf{/docs:} Contiene la documentación del proyecto. 
\end{itemize}

Dentro del proyecto Laravel los directorios más importantes son:

\begin{itemize}
    \item \textbf{/app:} Contiene el script para generar la base de datos sintética.
    \begin{itemize}
        \item \textbf{/Exports:} Contiene los ficheros necesarios para realizar la exportación de la base de datos a el fichero Excel.
        \item \textbf{/Http:} 
        \begin{itemize}
            \item \textbf{/Controllers:}
            \item \textbf{/Middleware:}
        \end{itemize}
        \item \textbf{/JoinsTablas:} Contiene el fichero que permite devuelve un outter join entre todas las tablas.
        \item \textbf{/Models:} Contiene los modelos de las tablas de la base de datos, cada uno con todos sus atributos y con sus relaciones.
        \item \textbf{/Rules:} Contiene las reglas que hemos añadido para la validación de datos como por ejemplo para comprobar correo único en la base de datos.
        \item \textbf{/Utilidades:} Contiene el fichero que permite tanto la encriptación de datos como la desencriptación .
    \end{itemize}
    \item \textbf{/config:} Contiene los ficheros de configuración del proyecto Laravel.
    \begin{itemize}
        \item \textbf{/app.php:} Fichero de configuración generales del proyecto, como por ejemplo nombre de la aplicación, zona horaria, etc.
        \item \textbf{/auth.php} Fichero de configuración de la autentificación del login.
    \end{itemize}
    \item \textbf{/database:} Contiene los ficheros para poner la base de datos a punto para poder usar la aplicación.
    \begin{itemize}
        \item \textbf{/migrations:} Contiene ficheros para poder crear la estructura de cada una de las tablas, con sus atributos y sus relaciones.
        \item \textbf{/seeders:} Contiene ficheros para introducir los datos necesarios para que funcione la aplicación.
    \end{itemize}
    \item \textbf{/public:} 
    \begin{itemize}
        \item \textbf{/css:} Contiene los ficheros \textit{.css} que van a servir para dar formato a nuestra web.
        \item \textbf{/js:} Contiene los ficheros \textit{.js} los cuales va a añadir animaciones y efectos a nuestra web.
    \end{itemize}
    \item \textbf{/resources/views:} Contiene los ficheros HTML de todas las vistas de la web.
    \item \textbf{/routes/web.php:} Contiene todas las rutas de la web con sus correspondientes peticiones HTTP-
    \item \textbf{/tests:} Contiene todos los test realizados con phpunit.
    \begin{itemize}
        \item \textbf{/Featute:} Contiene los test realizados de la gestión de pacientes y usuarios.
        \item \textbf{/Ini:} Contiene ficheros para crear pacientes y usuarios para los tests.
    \end{itemize}
    \item \textbf{/.env:} Contiene las variables de entorno.
    \item \textbf{/phpunit.xml:} Contiene la configuración para poder ejecutar los tests.
\end{itemize}

\section{Manual del programador} \label{manual}

En este apartado se van a comentar todos los programas y versiones que van a ser necesarias para correr la aplicación en local.

\subsection{WAMP}

Es un conjunto de programas para permitir el desarrollo de aplicaciones web en local en sistemas Windows, esta consta de 4 herramientas principales:
\begin{itemize}
    \item \textbf{Apache:} Servidor web HTTP de código abierto.
    \item \textbf{MySQL:} Base de datos relacional 
    \item \textbf{PHP:} Lenguaje de programación centrado en el desarrollo web.
    \item \textbf{PhpMyAdmin:} Programa para gestionar bases de datos.
\end{itemize}

WAMP se descarga en la web \href{https://www.wampserver.com/}{https://www.wampserver.com/}, en el instalador usar todos los valores por defecto y la versión que se esta usando es la 3.2.3.3.

\subsection{composer}

Es un gestor de paquetes para PHP y es el que en nuestro proyecto nos ha permitido tanto descargar Laravel como las librerías de PHP que hemos necesitado. La versión que se esta usando es la 2.0.13.

Composer se descarga en la web \href{https://getcomposer.org/}{https://getcomposer.org/}, en el instalador usar todos los valores por defecto, excepto cuando pregunta por la version de PHP que queremos usar, en este caso seleccionaremos "\textit{C:\textbackslash{}wamp64\textbackslash{}bin\textbackslash{}php\\\textbackslash{}php7.4.9\textbackslash{}php.exe}" como podemos ver en la imagen \ref{fig:composer}.

\imagenTam{composer}{Versión de PHP a elegir en el instalador de composer}{0.7}

\subsection{Python 3}

Es un lenguaje de programación, el cual se ha usado para la creación de la base de datos sintética, y es necesario tener Python instalado en el ordenador para permitir correr este script. La versión que se esta usando es la 3.9.5.

Python se descarga en la web \href{https://www.python.org/downloads/release/python-395/}{https://www.python.org/downloads/release\\/python-395/}, en el instalador seleccionar las opciones por defecto, excepto en la localización que vamos a poner "\textit{C:\textbackslash{}Python3.9\textbackslash{}}" como podemos ver en la imagen \ref{fig:python}.

\newpage

\subsubsection{Librerías Python}

Para este proyecto va a ser necesario tener instaladas la siguiente librerías de Python:
\begin{itemize}
    \item numpy
    \item mysql.connector
    \item faker
    \item datetime
\end{itemize}

Para instalar las siguiente librerías podemos usar el comando "\textit{pip install nombre de la librería}"

\imagenTam{python}{Localización a elegir en el instalador de python}{0.7}

\section{Compilación, instalación y ejecución del proyecto}

En este apartado se va a explicar como instalar y ejecutar el proyecto en un entorno local.

\subsection{Importar el proyecto}

Una vez tengamos instalado todos los programas y librerías como se ha explicado en el punto \ref{manual}, lo primero vamos a hacer va a ser importar el proyecto, para esto seguiremos los siguientes pasos:

\begin{enumerate}
    \item Acceder al repositorio: \href{https://github.com/CarlosOrtu/TFG_HUBU}{https://github.com/CarlosOrtu/TFG\_HUBU}.
    \item Descargar el contenido del repositorio desde \textbf{Code > Download ZIP} como se puede ver en la imagen \ref{fig:git_descargar}
    \imagenTam{git_descargar}{Descargar el proyecto GitHub}{0.5}
    \item Descomprimir el proyecto.
\end{enumerate}
    
\subsection{Añadir el proyecto a WAMP}

Una vez tenemos todos los archivos del proyecto, lo siguiente que tendremos que hacer será añadir este proyecto a WAMP. Para esto seguiremos los siguientes pasos.

\begin{enumerate}
    \item Accedemos a la ruta "\textit{C:\textbackslash{}wamp64\textbackslash{}www}"{} o a la ruta donde se haya descargado WAMP y posteriormente a la carpeta www.
    \item Creamos una carpeta con el nombre TFG\_HUBU.
    \item Pegamos todos los ficheros que hemos descomprimido del proyecto dentro de esta carpeta.
\end{enumerate}

\subsection{Instalar dependencias}

Una vez que tenemos el proyecto ya añadido a WAMP nos faltaría añadir todas las dependencias, es decir todas las librerías de PHP que se van a instalar con un solo comando gracias al fichero \textit{composer.json}. Para esto seguiremos los siguientes pasos:

\begin{enumerate}
    \item Abrimos el terminal de windows.
    \item Accedemos al directorio "\textit{cd C:\textbackslash{}wamp64\textbackslash{}www\textbackslash{}TFG\_HUBU\textbackslash{}Laravel}"
    \item Ejecutamos el comando \textit{composer install}.
\end{enumerate}

\subsection{Poner a punto las variables de entorno}

Para el correcto funcionamiento de la aplicación vamos a definir las variables de entorno, las cuales se encargan, entre otras cosas, de establecer la conexión con la base de datos. Para esto seguiremos los siguientes pasos:

\begin{enumerate}
    \item Accedemos al directorio "\textit{cd C:\textbackslash{}wamp64\textbackslash{}www\textbackslash{}TFG\_HUBU\textbackslash{}Laravel}"
    \item Creamos el fichero donde se definen las variables de entorno con el comando "\textit{copy .env.example .env}.
    \item Abrimos el fichero .env con cualquier editor de texto.
    \item Cambiamos las variables de entorno:
    \begin{itemize}
        \item APP\_NAME=TFG\_HUBU
        \item DB\_DATABASE=hubu
    \end{itemize}
    \item Generamos la clave automaticamente con el comando "\textit{php artisan key:generate}"
\end{enumerate}

\subsection{Crear la base de datos}

Lo primero que tendremos que hacer será ejecutar wampserver para tener disponibles todas las herramientas. Una vez ejecutado, vamos a crear una base de datos con el nombre \textbf{hubu} como hemos definido anteriormente en las variables de entorno. Esto lo podemos realizar de dos maneras:

\begin{enumerate}
    \item Mediante la consola MySQL:
    \begin{enumerate}
        \item Para acceder a la consola MySQL seleccionamos el icono de wamp con el clic izquierdo, en el menú seleccionamos MySQL y por último seleccionamos MySQL console como podemos ver en la imagen \ref{fig:mysql_consola}.
        \imagenTam{mysql_consola}{Acceder a la consola MySQL}{0.8}
        \item Una vez en la consola ingresamos el usuario \textbf{root} y la contraseña la dejamos vacía.
        \item Con la sesión ya iniciada ejecutamos el comando "\textit{CREATE DATABASE memoria;}".
    \end{enumerate}
    \item Mediante PhpMyAdmin:
    \begin{enumerate}
        \item Accedemos en el navegador a la url "\textit{http://localhost/phpmyadmin/}".
        \item Iniciamos sesión con el usuario \textbf{root} y la contraseña vacía.
        \item Con la sesión ya iniciada seleccionamos \textbf{Nueva} en el menú (\ref{fig:phpmyadmin_crear}) y le ponemos como nombre a la base de datos \textbf{hubu}.
        \imagenTam{phpmyadmin_crear}{Acceder a la consola MySQL}{0.3}
    \end{enumerate}
\end{enumerate}

\subsection{Poner a punto la base de datos}

Una vez creada la base de datos necesitamos crear la estructura y añadir los datos necesarios para que funcione como los roles y el primer usuario para poder realizar el login. Esto lo podemos realizar de dos maneras:

\begin{enumerate}
    \item Mediante \textit{migrations} y \textit{seeders} de Laravel:
    \begin{enumerate}
        \item Abrimos la terminal de windows, accedemos a la ruta "\textit{C:\textbackslash{}wamp\\64\textbackslash{}www\textbackslash{}TFG\_HUBU\textbackslash{}Laravel}"
        \item Ejecutamos el comando "\textit{php artisan migrate}" para ejecutar las migraciones y crear la estructura de tablas.
        \item Ejecutamos el comando "\textit{php artisan db:seed}" para ejecutar los \textit{seeders} y añadir los primeros datos.
    \end{enumerate}
    \item Mediante scripts SQL:
    \begin{enumerate}
        \item Abrimos la consola de MySQL como hemos explicado en el punto anterior.
        \item Una vez en la consola ingresamos el usuario \textbf{root} y la contraseña la dejamos vacía.
        \item Ejecutamos el comando "\textit{use hubu}" para que los scripts se ejecuten sobre esta base de datos.
        \item Ejecutamos los 3 scripts con los siguientes comandos:
        \begin{itemize}
            \item "\textit{source C:/wamp64/www/TFG\_HUBU/SQL/create\_users\_roles.sql;}"
            \item "\textit{source C:/wamp64/www/TFG\_HUBU/SQL/create\_pacientes.sql;}"
            \item "\textit{source C:/wamp64/www/TFG\_HUBU/SQL/add\_roles\_admin.sql;}"
        \end{itemize}
    \end{enumerate}
\end{enumerate}

\subsection{Acceso a la web}

Una vez se han realizado todos estos pasos ya podemos acceder a la web. Para esto seguiremos los siguientes pasos:

\begin{enumerate}
    \item Accedemos a la url "\textit{http://localhost/TFG\_HUBU/Laravel/public}"
    \item Iniciamos sesión con el usuario creado cuyo correo es \textbf{administrador@gmail.com} y contraseña \textbf{1234}.
\end{enumerate}

\section{Pruebas del sistema}

Como ya se ha comentado en la memoria para este proyecto se han realizado tres tipos de pruebas: pruebas de integración, pruebas de validación y pruebas de seguridad. Como las pruebas de validación son pruebas manuales realizadas tanto por el cliente como por el usuario final no va a existir ninguna manera de ejecutarlas automáticamente como las otras dos.

\subsection{Pruebas de integración}

Las pruebas de integración se han realizado para asegurar que la introducción, la modificación y la eliminación tanto de pacientes como de usuarios funciona sin ningún problema.

Para ejecutar estas pruebas primer se han tenido que realizar todos los pasos indicados en el apartado \ref{manual}, luego se tendrán que seguir los siguientes pasos:

\begin{enumerate}
    \item Abrir el terminal de windows.
    \item Acceder a la ruta "\textit{C:\textbackslash{}wamp64\textbackslash{}www\textbackslash{}TFG\_HUBU\textbackslash{}Laravel}
    \item Ejecutar el comando "\textit{php artisan test --testsuite=Feature}"
\end{enumerate}

\newpage

Si todo se ha realizado correctamente el resultado de los test será el de la imagen \ref{fig:test_integracion}.

\imagenTam{test_integracion}{Resultado de los test de integración}{0.7}

\subsection{Pruebas de seguridad}

Estas pruebas se han realizado para asegurar el correcto funcionamiento de los \textit{middleware} y asegurar que usuarios sin permiso no puedan acceder a rutas restringidas.

Para la ejecución de las pruebas se han de seguir los siguientes pasos:

\begin{enumerate}
    \item Abrir el terminal de windows.
    \item Acceder a la ruta "\textit{C:\textbackslash{}wamp64\textbackslash{}www\textbackslash{}TFG\_HUBU\textbackslash{}Laravel}
    \item Ejecutar el comando "\textit{php artisan test --testsuite=Seguridad}"
\end{enumerate}

Si todo se ha realizado correctamente el resultado de los test será el de la imagen.

\textcolor{red}{Faltan por implementar}