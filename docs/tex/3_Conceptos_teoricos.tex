\capitulo{3}{Conceptos teóricos}

En este apartado se van a tratar los conceptos teóricos más importantes del proyecto, empezando con la arquitectura del framework de back-end, siguiendo con la seguridad y acabando con los conceptos estadíticos. 

\section{Arquitectura MVC}\label{MVC}

La arquitectura Modelo-Vista-Controlador sirve para separar el código dependiendo de la funcionalidad de este, así conseguir lograr 3 capas con un trabajo especifico en cada una de ellas\cite{MVC}. Actualmente algunos frameworks como Laravel integran este tipo de arquitectura pero añadiendo capas extra como el enrutado y los middleware como podemos ver en la imagen \ref{fig:MVC_Laravel}.

\imagenTam{MVC_Laravel}{Arquitectura MVC Laravel}{0.8}

\subsection{Modelo}

Es la capa que se encarga de interactuar con los datos, por lo tanto tendrá que tener la capacidad de añadir, modificar y eliminar los datos que sean necesarios. Lo normal suele ser usar alguna librería o algún mecanismo que incluya el framework para no tener que trabajar directamente con consultas SQL y permitir trabajar con objetos y clases lo que supone una abstracción de base de datos y persistencia en objetos\cite{MVC}.

\subsection{Vista}

Las vistas son la capa que va a servir como interfaz al usuario y le va a permitir interactuar y visualizar los datos sin trabajar directamente con ellos ya que de esto se encargan las otras dos capas. Las vistas suelen estar escritas en HTML y CSS principalmente, pudiendo incluir otros lenguajes como JavaScript o PHP e incluso procesadores de plantillas como Blade Templates de Laravel \cite{MVC}.

\subsection{Controlador}

Los controladores son la capa encargada de implementar la lógica de negocio que va a actuar como enlace entre las acciones que realice el usuario en la vista y los modelos. Pueden estar escritos en muchos lenguajes como PHP, Python, NodeJS, Java, C\# o Ruby \cite{MVC}.

\section{Encriptación simétrica} \label{simetrica}

La encriptación de datos es un método mediante el cual un conjunto de caracteres se modifican para convertirlos en otros completamente diferentes de los cuales no se podrá volver a obtener el conjunto original a no ser que se disponga de la clave o las claves necesarias. Existen dos tipos diferentes de encriptación: simétrica y asimétrica\cite{encriptacion}. 

La encriptación simétrica se basa en lograr tanto el cifrado como el descifrado mediante el uso de una sola clave, por lo tanto la seguridad de este método depende en la capacidad que se va a tener para que una persona no autorizada no tenga acceso a esta clave\cite{simetrico}. Estos algoritmos tienen dos tipos dependiendo de la división que se haga del mensaje original, cifrado de bloque y cifrado de flujo\cite{tiposcifrado}.

\subsection{Cifrado de bloque}
En el cifrado de bloque el mensaje original se divide en bloques de un tamaño fijo de bits, normalmente suelen ser de un tamaño 64 o 128 bits, estos bloques se encriptan de manera fija para todos sus elementos. El inconveniente de este tipo de cifrado es que si existen secuencias repetidas de caracteres en el texto original, también existirán estos caracteres repetidos en el texto encriptado\cite{cifradobloque}. Existen diferentes esquemas de cifrado dentro de los cifrados de bloque, los más importantes son DES, 3-DES, RC2, RC5, RC6 y AES\cite{simetrico}.

\subsubsection{AES}
El algoritmo AES (\textit{Advanced Encryption Standar}) agrupa el mensaje original en bloques de longitud variable, aunque el estándar fija el tamaño en 128 bits y la representación de estos bloques sería una matriz de 4x4 bytes. Lo mismo ocurre con la clave, tiene tres posibles longitudes 128, 192 y 256 y en el caso del AES-128, la contraseña se almacena de la misma manera que cada segmento del mensaje, en una matriz de 4x4. A parte de esta clave inicial, se generan un conjunto de 10 subclaves, una por cada iteración del algoritmo. Cada una de estar iteraciones realiza 4 operaciones básicas SubBytes, ShiftRows, MixColumns y AddRoundKey y para descifrar se realizan las mismas iteraciones y los mismos algoritmos pero de forma inversa\cite{AES}.

\section{Estadística} \label{estadistica}

\subsection{Conceptos estadísticos}
\textbf{Media:} Es una medida de tendencia que nos indica el punto medio de un conjunto de datos\cite{media}.

\begin{equation} 
Media(x) = \overline{x} = \frac{\sum_{i=1}^{n}x_{i}}{n}
\end{equation}

\textbf{Varianza:} Es una medida de dispersión que nos indica la distancia de un conjunto de datos a su media\cite{varianza}.

\begin{equation} 
Varianza(x) =  \sigma = \sqrt{\frac{\sum_{i=1}^{n}(x_{i}-\overline{x})^{2}}{n-1}}
\end{equation}

\textbf{Desviación:} Es la varianza al cuadrado\cite{varianza}.

\begin{equation} 
Desviación(x) =  \sigma^{2} = \frac{\sum_{i=1}^{n}(x_{i}-\overline{x})^{2}}{n-1}
\end{equation}

\textbf{Percentil:} Es una medida de posición que indica el porcentaje de datos que están por debajo de un valor\cite{percentil}. \label{percentil}

\textbf{Oblicuidad:} Es una medida que nos indica que tan asimétrica es una distribución respecto a su media\cite{oblicuidad}.

\begin{equation} 
Oblicuidad(x) = \frac{\sum_{i=1}^{n}(x_{i}-\overline{x})^3}{(n-1)\cdot S^3}
\end{equation}

\textbf{Curtosis:} Es una medida que según aumenta su valor nos indica que existe una concentración de valores tanto cerca de la media como muy lejos de ella, al tiempo que en los valores intermedios disminuye esta concentración\cite{curtosis}. 

\begin{equation} 
Curtosis(x) = \frac{\sum_{i=1}^{n}(x_{i}-\overline{x})^4}{(n-1)\cdot S^4}
\end{equation}
	
\textbf{Corrección de Bessel:} La corrección de Bessel se usa para corregir el sesgo estadístico cuando no se esta trabajando con la población entera sino que se esta trabajando con una muestra y consiste en sustituir $n$ en el divisor por n-1, como he realizado en todos los conceptos anteriormente explicados\cite{bessel}.


\subsection{Distribuciones de probabilidad}

Las distribuciones son funciones que indican la probabilidad que ocurra cierto suceso. Según sea el tipo de variable continua o discreta se elegirá un tipo de distribución u otro.

\textbf{Distribución normal:} Es una distribución de valores continuos definida por dos parámetros, la media $\mu$ y la desviación $\sigma$, la función de densidad es simétrica respecto a su media y dependiendo de la desviación va a tener una anchura determinada\cite{distribuciones}.

\imagen{distribucion_normal}{Gráfica de densidad de la distribución normal}

\textbf{Distribución de poisson:} Es una distribución de valores discretos la cual está definida por un parámetro $\lambda$, y este parámetro representará el valor que es más probable de obtener dentro de la gráfica.

\imagen{distribucion_poisson}{Tres gráficas de la distribución de poisson según $\lambda$}

\textbf{Distribución geométrica:} Es una distribución de valores discretos la cual esta definida por un parámetro $p$, y este parámetro indica la probabilidad de obtener el resultado.

\imagen{distribucion_geometrica}{Gráfica de una distribución geométrica}
