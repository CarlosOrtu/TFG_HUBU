\apendice{Especificación de diseño}

\section{Introducción}

En este anexo se van a explicar los diferentes aspectos de diseño que se ha llevado a cabo para la realización del proyecto. Dentro de este apartado vamos a incluir el diseño de datos, el diseño procedimental y el diseño arquitectónico.

\section{Diseño de datos}

Se puede dividir la parte de diseño de datos del proyecto en tres partes principales:
\begin{itemize}
    \item \textbf{Diagramas de tablas relacionadas con usuarios:} Van a representar la estructura de datos que se ha llevado a cabo para establecer el login de los usuarios dentro de la aplicación.
    \item \textbf{Diagramas de tablas relacionadas con pacientes:} Van a representar la estructura de datos que se ha llevado a cabo para almacenar todos los datos de los pacientes.
    \item \textbf{Diccionario de datos relacionados con pacientes:} Va a contener: una pequeña descripción de cada uno de los datos, el tipo de dato como los posibles opciones y posibles combinaciones entre datos.
\end{itemize}

\newpage

\subsection{Diagramas de tablas relacionadas con usuarios}

\imagen{diagrama_ER_usuarios}{Diagrama E/R de los usuarios}

\imagen{diagrama_relacional_usuarios}{Diagrama relacional de los usuarios}

\newpage

\subsection{Diagramas de tablas relacionadas con pacientes}

\imagenTam{diagrama_ER_pacientes}{Diagrama E/R de los pacientes}{1}

\imagenTam{diagrama_relacional_pacientes}{Diagrama relacional de los pacientes}{1}

\subsection{Diccionario de datos}

\subsubsection{Pacientes}

\textbf{Descripción:} Esta tabla va a constar de todos los pacientes que sean registrados dentro de la web.

\begin{longtable}{|c |c |p{4cm} |p{3cm} |}
\hline
Campo & Tipo de dato & Descripción & Posibles opciones\\ \hline
id\_paciente & int & Número identificador único de cada paciente &\\ \hline
nombre & varchar & Nombre del paciente &\\ \hline
apellidos & varchar & Dos primeros apellidos del paciente &\\ \hline
sexo & varchar & Sexo del paciente & Varón o mujer\\  \hline
nacimiento & date & Fecha de nacimiento del paciente &\\ \hline
raza & varchar & Raza a la que pertenece el paciente & Caucásico, asiático, africano o latino\\ \hline
profesión & varchar & Profesión en la que ha trabajado el paciente & Construcción, minería, pintor, peluquero, industria textil, mecánico, limpieza, cerámicas, otras o desconocido. \\ \hline
fumador & varchar & Si el paciente es fumador o no & Fumador, exfumador o nunca fumador\\ \hline
num\_tabaco\_dia & int & Número de cigarrillos al día & \\ \hline
bebedor & varchar & Si el paciente consume alcohol o no & Bebedor activo, exbebedor, nunca bebedor\\ \hline
carcinógenos & varchar & Si el paciente ha estado expuesto a carcinógenos & Asbesto u otro  \\\hline
ultima\_modificacion & date & Fecha en la que se ha realizado la última modificación al paciente &  \\\hline
\end{longtable}

\subsubsection{Enfermedades}

\textbf{Descripción:} Esta tabla va a constar de los datos relacionados con el cancer de pulmón de cada uno de los pacientes.

\begin{longtable}{|c |c |p{3.3cm} |p{3cm} |}
\hline
Campo & Tipo de dato & Descripción & Posibles opciones\\ \hline
id\_enfermedad & int & Número identificador unico del cancer de pulmón del paciente & \\\hline
id\_paciente & int & Número identificador del paciente que sufre el cancer de pulmón &\\\hline
fecha\_primera\_consulta & date & Fecha de la primera consulta en la que se detecto el cancer &\\\hline
fecha\_diagnostico & date & Fecha en la que se diagnostico el cancer de pulmón &\\\hline
ECOG & int & Escala en la cual medir la calidad de vida del paciente & Del 0 al 4\\\hline
T & int & Tamaño y extensión del tumor principal & Del 1 al 4\\\hline
T\_tamano & int & Tamaño en mm del tumor principal & \\\hline
N & int & Extensión de cáncer que se ha diseminado a los ganglios & Del 1 al 3 \\\hline
N\_afectación & varchar & Tipo de afectación que ha tenido sobre los ganglios &Uni ganglionar o afectación multiestación\\\hline
M & varchar & Valor para determinar si el cáncer se ha metastatizado & 0, 1a, 1b ó 1c\\\hline
num\_afec\_metas & varchar & Número de localizaciones metástasicas & 0, 1, 2-4 ó mayor que 4\\\hline
TNM  & varchar  & Manera de clasificar los tipos de cáncer según los valores de las anteriores tablas T, N y M & IA1, IA2, IA3, IB, IIA, IIB, IIIA, IIIB, IIIC, IVa ó IVb \\\hline
tipo\_muestra & varchar & Tipo de muestra que se ha extraido & citología, biopsia ó bloque celular desde citología\\\hline
histología\_tipo & varchar & Tipo de histología que se ha realizado & Adenocarcinoma, epidermoide, adenoescamoso, carcinoma de células grandes, sarcomatoide ó indiferenciado\\\hline
histología\_subtipo & varchar & Subtipo de la histología que se ha realizado & Desconocido, acinar, lepídico, papilar, micropapilar, sólido, mucinoso, células claras u otro.\\\hline
histología\_subtipo & varchar & Subtipo de la histología que se ha realizado & esconocido, acinar, lepídico, papilar, micropapilar, sólido, mucinoso, células claras u otro.\\\hline
histología\_grado & varchar & Grado de la histología que se ha realizado & Bien diferenciado, moderadamente diferenciado, mal diferenciado o no especificado.\\ \hline
tratamiento\_dirigido & boolean & Si el paciente ha recibido algun tratamiento dirigido & Si o no\\ \hline
\end{longtable}

\subsubsection{Metástasis}

\textbf{Descripción:} Esta tabla va a constar de las diferentes metástasis que genere un cáncer de pulmón.

\begin{longtable}{|c |c |p{4.8cm} |p{3.2cm} |}
\hline
Campo & Tipo de dato & Descripción & Posibles opciones\\ \hline
id\_metastasis & int & Número identificador único de la metástasis & \\\hline
id\_enfermedad & int & Número identificador del cáncer de pulmón que ha provocado la metástasis&\\\hline
localización & varchar & Localización de la metastasis & Pulmón contralateral, implantes pleurales, derrame pleural, hígado, hueso, suprarrenal, renal, SNC, derrame pericárdico, carcinomatosis meníngea, linfangitis pulmonar carcinomatosa, adenopatías supradiafragmáticas extratorácicas, adenopatías infradiafragmáticas, páncreas, peritoneo, cutánea, muscular, tejidos blandos u otra \\ \hline
\end{longtable}

\subsubsection{Síntomas}

\textbf{Descripción:} Esta tabla va a constar de los diferentes síntomas que ha presentado el paciente relacionado con el cáncer de pulmón.

\begin{longtable}{|c |c |p{4.8cm} |p{3.2cm} |}
\hline
Campo & Tipo de dato & Descripción & Posibles opciones\\ \hline
id\_síntomas &  int &  Número identificador único del síntoma &\\\hline
id\_enfermedad & int & Número identificador del cáncer de pulmón que ha provocado este síntoma & \\\hline
tipo & varchar & Tipo de síntoma & Asintomático, tos, disnea, pérdida de peso, anorexia, aumento de expectoración, hemoptisis, dolor torácico, dolor otra localización, clínica neurológica, fractura patológica, otros u desconocido \\\hline
fecha\_inicio & date & Fecha de inicio del síntoma &\\ \hline
\end{longtable}


\subsubsection{Biomarcadores}

\textbf{Descripción:} Esta tabla va a constar de los diferentes biomarcadores que se han realizado sobre el cáncer de pulmón de un paciente. 

\begin{longtable}{|c |c |p{4.7cm} |p{3.1cm} |}
\hline
Campo & Tipo de dato & Descripción & Posibles opciones\\ \hline
id\_biomarcador & int &Número identificador único del biomarcador &\\ \hline
id\_enfermedad & int & Número identificador del cáncer de pulmón sobre el que se ha realizado este biomarcador &\\ \hline
nombre & varchar & Tipo de biomarcador & NGS, PDL1, EGFR, ALK, ROS1, KRAS, BRAF, HER2, NTRK, FGFR1, RET, MET, PL3K, TMB u otros\\ \hline
tipo & varchar &Subtipo de biomarcador & Biopsia sólida, biopsia líquidam, en célula tumorales con IHQ, score combinado/ CPS, exón 19, exón 21, exón 20, exón 18, T790M positivo, T790M negativo, traslocado, no traslocado, mutación, Clase I, Clase II, Clase III, amplificado, no alterado, reordenado, BRCA1, BRCA2, KIT, JAK, STK11, ARID1A, smoothedend u otros    \\\hline
sub\_tipo & varchar &  Diferentes valores que puede tomar el biomarcador & Oncomine Focus, Oncodeep; Guardant 360, Focus DX, FoliguidLDX, $\leq 1\%, 1\%-49\%, \geq 49\%$, fusión EML4, KRASG12C, KRASG12V, BRAFV600E, NTRK 1, NTRK 2, NTRK 3, FGFR 1, FGFR 2, FGFR 3 u otros\\ \hline
\end{longtable}

En la siguiente tabla indicaré que posibilidades de tipos, subtipos y valores pueden existir dentro de la base de datos.

\begin{longtable}{|c |p{7.2cm} |p{4.7cm} |}
\hline
Nombre & Tipo & Subtipo\\ \hline
NGS & Biopsia sólida o Biopsia líquida & Oncomine Focus, Oncodeep, Guardant 360, Focus DX o FoliguidLDX\\ \hline
PDL1 & En célula tumorales con IHQ o Score combinado/ CPS & $\leq 1\%, 1\%-49\%, \geq 49\%$ \\ \hline
EGFR & Exón 19, exón 21, exón 20, exón 18, T790M positivo, T790M negativo u otras & \\ \hline
ALK & Traslocado, No traslocado o Mutación & Si es taslocado: Fusión EML4 u Otra   \\ \hline
ROS1 & Traslocado o No traslocado & \\ \hline
KRAS & Nutado o No mutado & Si es mutado: KRASG12C, KRASG12 u Otras\\ \hline
BRAF & Mutado, No mutado. Clase I, Clase II o Clase III & Si es mutado: BRAFV600E u Otra \\ \hline
HER2 & Mutado, Amplificado o No alterado & \\ \hline
NTRK & mutado, no mutado, reordenado & NTRK 1, NTRK 2 o NTRK 3 \\ \hline
FGFR1 & Amplificado, No amplificado o Mutación &  \\ \hline
RET & Traslocado, No traslocado o Mutación &  \\ \hline
MET & Mutación Exón14, No mutado, Amplificado o Sobreexpresado & \\ \hline
PI3K & Mutado o No mutado &  \\ \hline
TMB & Alto o Bajo &  \\ \hline 
Otro & BRCA1, BRCA2, KIT, JAK, STK11, ARID1A, Smoothedend u Otros &  \\ \hline 
\end{longtable}

\subsubsection{Pruebas realizadas}

\textbf{Descripción:} Esta tabla va a constar de las diferentes pruebas que se han realizado sobre el cáncer del paciente.

\begin{longtable}{|c |c |p{4.7cm} |p{3.1cm} |}
\hline
Campo & Tipo de dato & Descripción & Posibles opciones\\ \hline
id\_prueba & int & Número identificador único de la prueba &\\\hline
id\_enfermedad & int & Número identificador del cáncer de pulmón sobre el que se ha realizado esta prueba & \\ \hline
tipo & varchar & Tipo de prueba que se ha realizado & Radiografía de tórax, TAC TAP, TAC SNC, PET-TAC, GGO, RMN SNC, RMN body u otras \\ \hline
\end{longtable}


\subsubsection{Técnicas realizadas}

\textbf{Descripción:} Esta tabla va a constar de las diferentes técnicas que se han realizado sobre el cáncer del paciente.

\begin{longtable}{|c |c |p{4.7cm} |p{3.1cm} |}
\hline
Campo & Tipo de dato & Descripción & Posibles opciones\\ \hline
id\_técnica & int &  Número identificador único de la técnica & \\\hline
id\_enfermedad & int & Número identificador del cáncer de pulmón sobre el que se ha realizado esta técnica & \\\hline
tipo & varchar & Tipo de técnica que se ha realizado sobre el cáncer de pulmón & Broncoscopia, EBUS, mediastinoscopia, BAG pulmonar, BAG extrapulmonar, cirugía diagnóstico-terapéutica u otra.\\ \hline
\end{longtable}

\subsubsection{Otros tumores}

\textbf{Descripción:} Esta tabla va a constar de los diferente tumores que pueda tener el paciente a parte del cáncer de pulmón. 

\begin{longtable}{|c |c |p{4.7cm} |p{3.1cm} |}
\hline
Campo & Tipo de dato & Descripción & Posibles opciones\\ \hline
id\_tumor & int & Número identificador único del tumor  & \\\hline
id\_enfermedad & int &Número identificador del cancer de pulmon del paciente que tiene otro tumor & \\\hline
tipo & varchar & Tipo de cáncer que se ha detectado & Pulmón, ORL, vejiga, renal, colon, mama, páncreas, esofagogástrico, próstata, ginecológico, hígado, linfático, SNC u otro\\ \hline
\end{longtable}

\subsubsection{Antecedentes oncológicos}

\textbf{Descripción:} Esta tabla va a constar de los diferente canceres que haya tenido el paciente anteriormente. 

\begin{longtable}{|c |c |p{4.2cm} |p{3cm} |}
\hline
Campo & Tipo de dato & Descripción & Posibles opciones\\ \hline
id\_antecedente\_o & int & Número identificador único del antecedente oncológico &\\\hline
id\_paciente & int & Número identificador del paciente que ha tenido este antecedente oncológico &\\\hline
tipo & varchar & Tipo de cáncer que ha sufrido el paciente con anterioridad & Pulmón, ORL, vejiga, renal, colon, mama, páncreas, esofagogástrico, próstata, hígado, ginecológico, linfático, SNC u otro.\\ \hline
\end{longtable}

\subsubsection{Antecedentes familiares}

\textbf{Descripción:} Esta tabla va a constar de los diferente familiares del paciente que hayan sufrido algún tipo de cáncer.

\begin{longtable}{|c |c |p{4.4cm} |p{3cm} |}
\hline
Campo & Tipo de dato & Descripción & Posibles opciones\\ \hline
id\_antecedente\_f &  int & Número identificador único del antecedente familiar & \\\hline
id\_paciente & int & Número identificador del paciente que ha tenido este antecedente familiar & \\\hline
familiar & varchar & Nombre del parentesco que se tiene con el familiar que ha sufrido cáncer & \\ \hline
\end{longtable}

\subsubsection{Enfermedades familiar}

\textbf{Descripción:} Esta tabla va a constar de los diferente enfermedades que ha sufrido el familiar del paciente con cáncer.

\begin{longtable}{|c |c |p{4.4cm} |p{3cm} |}
\hline
Campo & Tipo de dato & Descripción & Posibles opciones\\ \hline
id\_enfermedad\_f & int & Número identificador único de la enfermedad familiar & \\\hline
id\_antecedente\_f & int & Número identificador del familiar que ha tenido esta enfermedad & \\\hline
tipo & varchar & Tipo de enfermedad que ha sufrido & Pulmón, ORL, vejiga, renal, colon, mama, páncreas, esofagogástrico, próstata, ginecológico, hígado, linfático, SNC u otro. \\ \hline
\end{longtable}

\subsubsection{Reevaluaciones}

\textbf{Descripción:} Esta tabla va a constar de los diferentes reevaluaciones que se vayan realizando sobre el paciente.

\begin{longtable}{|c |c |p{3.4cm} |p{3cm} |}
\hline
Campo & Tipo de dato & Descripción & Posibles opciones\\ \hline
id\_reevaluacion & int & Número identificador único de la reevaluación & \\\hline
id\_paciente & int & Número identificador del paciente sobre el que se ha realizado la reevaluación &\\\hline
fecha & date & Fecha en la que se ha realizado la reevaluación al paciente&\\\hline
estado & varchar & Estado del paciente en la reevaluación & Sin evidencia de enfermedad/respuesta completa, respuesta parcial, enfermedad estable o progresión/recaída \\\hline
progresión\_localización & varchar & Localización de donde se esta llevando a cabo la progresión & Pulmón contralateral, implantes pleurales, derrame pleural, hígado, hueso, suprarrenal, renal, SNC, derrame pericárdico, carcinomatosis meníngea, linfangitis pulmonar carcinomatosa, adenopatías supradiafragmáticas extratorácicas, adenopatías infragmáticas, páncreas, peritoneo, cutánea, muscular, tejidos blandos u otra \\\hline
tipo\_tratamiento & varchar &Tipo de tratamiento que se esta llevando a cabo sobre el paciente en el momento de la reevaluación & Tratamiento activo o cuidados paliativos\\ \hline
\end{longtable}

\subsubsection{Tratamientos}

\textbf{Descripción:} Esta tabla va a constar de los diferentes tratamientos que vaya recibiendo sobre el paciente. 

\begin{longtable}{|c |c |p{4.6cm} |p{3.2cm} |}
\hline
Campo & Tipo de dato & Descripción & Posibles opciones\\ \hline
id\_tratamiento & int &  Número identificador único del tratamiento &\\\hline
id\_paciente & int & Número identificador del paciente que ha recibido el tratamiento &\\\hline
tipo & varchar & Tipo de tratamiento que se ha llevado a cabo & Cirugía, quimioterapia o	radioterapia\\\hline
subtipo & varchar & Subtipo de tratamiento que se ha llevado a cabo & Neumonectomía, lobectomía, bilobectomía, segmentectomía, resección atípica, neoadyuvancia, adyuvancia, enfermedad avanzada, radical o paliativa\\\hline
dosis & int & Dosis que se ha dado al paciente del tratamiento &\\\hline
localización & varchar & Localización de la zona donde se haya llevado a cabo el tratamiento& Pulmonar, pulmonar + mediastino,  ósea, suprarrenal, SNC, hígado, ganglionar u otro\\\hline
fecha\_inicio & date & Fecha de inicio del tratamiento&\\\hline
fecha\_fin & date & Fecha de fin del tratamiento&\\ \hline
\end{longtable}

En la siguiente tabla indicaré que posibilidades de tipos, subtipos y valores pueden existir dentro de la base de datos.

\begin{longtable}{|c |p{5cm} |p{5.7cm} |}
\hline
Tipo & Subtipo & Localización\\ \hline
Cirugía & Neumonectomía, lobectomía, bilobectomía, segmentectomía o resección atípica & \\ \hline
Quimioterapia & Neoadyuvancia, Adyuvanci o Enfermedad avanzada & \\ \hline
Radioterapia & Radical o paliativa & Pulmonar, pulmonar + mediastino,  ósea, suprarrenal, SNC, hígado, ganglionar u otro\\ \hline
\end{longtable}

\subsubsection{Intenciones}

\textbf{Descripción:} Esta tabla va a constar de las diferentes intenciones que tengan cada uno de los tratamientos.

\begin{longtable}{|c |c |p{2.6cm} |p{2.3cm} |}
\hline
Campo & Tipo de dato & Descripción & Posibles opciones\\ \hline
id\_intención & int & Número identificador único de la intención & \\\hline
id\_tratamiento & int &Número identificador del tratamiento que que ha sido realizado con esta intención  & \\\hline
ensayo & varchar & Determina si la intención ha sido para un ensayo clínico & Si o no\\\hline
ensayo\_fase & int & Fase en la que se encuentra el ensayo & Del 1 al 4\\\hline
tratamiento\_acceso\_expandido & boolean & Determina si la intención ha sido un tratamiento por acceso expandido & Si o no\\\hline
tratamiento\_fuera\_indicación & boolean & Determina si la intención ha sido un  tratamiento fuera de indicación & Si o no\\\hline
medicación\_extranjera & boolean & Determina si se ha usado medicación extranjera en esta intención & Si o no\\\hline
esquema & varchar & Esquema que se ha usado & Monoterapia, combinación, concomitancia con radioterapia o combinación \\\hline
modo\_administración & varchar & Modo en el cual se ha administrado los fármacos & Oral, intravenoso, intravenoso u otra\\ \hline
tipo\_farmaco & varchar & Tipo de fármaco que se ha administrado & Quimioterapia, inmunoterapia, tratamiento dirigido, antiangiogénico, quimoterapia + inmunoterapia, quimioterapia + tratamiento dirigido, quimioterapia + antiangiogénico u otro
\\\hline
numero\_ciclos & int & Número de ciclos que se ha realizado & Del 1 al 6 excepto en enfermedad avanzada que va del 1 al 100\\ \hline
\end{longtable}

\subsubsection{Fármacos}

\textbf{Descripción:} Esta tabla va a constar de los diferentes fármacos que se hayan usado en las intenciones.

\begin{longtable}{|c |c |p{4cm} |p{4.2cm} |}
\hline
Campo & Tipo de dato & Descripción & Posibles opciones\\ \hline
id\_fármaco &  int & Número identificador único de el fármaco &\\\hline
id\_intención & int & Número identificador de la intención en la cual ha sido usado este farmaco &\\\hline
tipo & varchar & Tipo de medicamento & Cisplatino, carboplatino, vinorelbina, paclitaxel, nab-paclitaxel, docetaxel, pemetrexed, gemcitabina, bevacizumab, ramucirumab, nintedanib, nivolumab, pembrolizumab, atezolizumab, avelumab, erlotinib, gefinitib, afatinib, dacomitinib, osimertinib, mobocertinib, amivantamab, crizotinib, alectinib,  brigatinib, ceritinib, lorlatinib, dabratinib, trametinib, tepotinib, capmatinib, tepotinib, trastuzumab-deruxtecán, fármaco en ensayo clínico / acceso expandido u otro\\ \hline
\end{longtable}

\subsubsection{Antecedentes médicos}

\textbf{Descripción:} Esta tabla va a constar de las diferentes enfermedades que haya tenido el paciente anteriormente. 

\begin{longtable}{|c |c |p{4.1cm} |p{3cm} |}
\hline
Campo & Tipo de dato & Descripción & Posibles opciones\\ \hline
id\_antecedente\_m & int & Número identificador único del antecedente médico& \\\hline
id\_paciente & int & Número identificador del paciente que ha tenido este antecedente médico & \\\hline
tipo\_antecedente & varchar & Tipo de enfermedad que ha sufrido & HTA, DM, DL, obesidad, EPOC, asma, IAM, ictus, enfermedad autoinmune, VIH, tuberculosis u otro\\ \hline
\end{longtable}

\subsubsection{Seguimientos}

\textbf{Descripción:} Esta tabla va a constar de los diferentes seguimientos que se han realizado al paciente. 

\begin{longtable}{|c |c |p{3.9cm} |p{3cm} |}
\hline
Campo & Tipo de dato & Descripción & Posibles opciones\\ \hline
id\_seguimiento & int & Número identificador único del seguimiento &\\\hline
id\_paciente & int & Número identificador del paciente al cual se le ha realizado el seguimiento &\\\hline
fecha & date & Fecha en la que se ha realizado el seguimiento & \\\hline
estado & varchar & Estado del paciente & Vivo sin enfermedad, vivo con enfermedad o fallecido \\\hline
fallecido\_motivo & varchar & Motivo del fallecimiento & Por enfermedad u otro motivo\\\hline
fecha\_fallecimiento & date & Fecha de fallecimiento del paciente &\\\hline
\end{longtable}

\subsubsection{Comentarios}

\textbf{Descripción:} Esta tabla va a constar de los diferentes comentarios que se han realizado sobre el paciente.

\begin{longtable}{|c |c |p{3.6cm} |p{4cm} |}
\hline
Campo & Tipo de dato & Descripción & Posibles opciones\\ \hline
id\_comentario & int & Número identificador único del seguimiento &\\\hline
id\_paciente & int & Número identificador del paciente al cual se le ha realizado el comentario &\\\hline
comentario & text & Comentario extra que se añade sobre el paciente& \\\hline
\end{longtable}

\section{Diseño procedimental}

\section{Diseño arquitectónico}

La arquitectura de la aplicación se ha basado en el estilo Modelo-Vista-Controlador (MVC) pero añadiendo más elementos como es el enrutado y los middleware, voy a realizar una breve explicación de cada uno de los elementos ya que esta arquitectura se explica más en detalle en los conceptos teóricos de la memoria.

\begin{itemize}
    \item \textbf{Modelo:} Se encarga de comunicarse con la base de datos de manera sencilla permitiendo realizar consultas y actualizaciones.
    \item \textbf{Vista:} Se encarga de permitir ver los datos al usuario mediante una interfaz.
    \item \textbf{Controlador:} Se encarga de implementar la lógica de negocio.
    \item \textbf{Enrutado:} Se encarga de recibir las peticiones HTTP y a partir de estas llamar al controlador correspondiente.
    \item \textbf{Middleware:} Se encarga de establecer una capa de seguridad a la hora de realizar las peticiones HTTP.
\end{itemize}

\imagen{MVC_diagrama}{Arquitectura MVC con enrutado y middleware}
