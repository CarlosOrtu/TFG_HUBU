\capitulo{2}{Objetivos del proyecto}

En este apartado se van a especificar los diferentes objetivos, tanto funcionales como técnicos como personales.

\section{Objetivos funcionales}

Los objetivos funcionales que se acordaron con los oncólogos y se han tratado de conseguir con este proyecto son:

\begin{itemize}
    \item Integrar un inicio de sesión para controlar el acceso a la aplicación web.
	\item Crear un sistema de gestión de usuarios en el cual los administradores puedan crear, modificar y eliminar usuarios.
	\item Realizar un sistema de gestión de datos personales donde se permita cambiar tanto la contraseña como el correo del propio usuario.
	\item Establecer un sistema de gestión de pacientes donde se pueda crear, eliminar y modificar pacientes y todos sus datos de una manera intuitiva y visual.
	\item Publicar la aplicación web en una plataforma en la nube para permitir a los oncólogos empezar a introducir datos.
	\item Facilitar la visualización y análisis de los datos con un sistema de gráficas y tablas.
	\item Automatizar la creación de una base de datos sintética con distintos tipos de distribuciones estadísticas para permitir realizar pruebas y comprobar el correcto funcionamiento de herramientas nombradas anteriormente como la visualización de los datos en gráficas.
\end{itemize}
\section{Objetivos técnicos}
Los objetivos técnicos que han sido necesario para llevar a cabo este proyecto son:

\begin{itemize}
    \item Usar \textit{Laravel} para el \textit{back-end} y todas las herramientas que este framework incluye.
    \item Emplear \textit{Bootstrap}, \textit{CSS}, \textit{HTML}, \textit{JavaScript} y \textit{jQuery} para el \textit{front-end} de la web y realizar el diseño de está lo mas funcional e intuitivo posible.
    \item Gestionar bases de datos \textit{MySQL} desde \textit{PhpMyAdmin}.
    \item Recordar el lenguaje \textit{SQL} para la realización de scripts que permitan crear la base de datos e introducir los datos necesarios para el correcto funcionamiento de la web.
    \item Usar la librería \textit{Google Charts} para la representación de las diferentes gráficas.
    \item Utilizar \textit{Python} y alguna librería de este lenguaje para la creación de la base de datos sintética.
    \item Usar \textit{GitHub} para permitir un control de versiones sobre el proyecto.
    \item Realizar el desarrollo del proyecto siguiendo la metodología \textit{SCRUM} sobre el proyecto y usar \textit{ZenHub} para permitir una mejor gestión de las tareas y del tiempo de cada una de estas.
\end{itemize}
\section{Objetivos personales}
\begin{itemize}
    \item Ayudar al departamento de oncología del Hospital Universitario de Burgos.
    \item Profundizar en los campos del desarrollo y diseño web.
    \item Poner en práctica todo lo aprendido durante estos cuatro años de carrera.
    \item Acabar mi ciclo formativo en la universidad con un proyecto funcional y que tenga una utilidad real.
    \item Aprender a usar \LaTeX{} para una mejorar la documentación de mis futuros proyectos.
\end{itemize}