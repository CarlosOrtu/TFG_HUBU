\capitulo{1}{Introducción}
El \textbf{cáncer} es un conjunto de enfermedades relacionadas las cuales son provocadas por una anomalía en las células que ocasiona que estas empiecen a dividirse continuamente y provoquen un mal funcionamiento de la zona incluso llegándose a expandir a otras partes del cuerpo, a esto se le llama \textbf{metástasis} \cite{cancerweb:online}. El \textbf{cáncer de pulmón} ocurre cuando este problema surge en el tejido pulmonar, este cáncer es el que más muertes provoca actualmente en el mundo, tanto en hombres como en mujeres, siendo el tabaquismo la principal causa de este problema llegando a suponer hasta el 80\% de los casos\cite{cancerpulmon:online}.

Existe gran variedad en cuanto a los diferentes tipos de \textbf{cáncer de pulmón}, los \textbf{tratamientos} que se llevan a cabo y las \textbf{herramientas de detección} de las que se hacen uso. Dicha pluralidad hace que pueda resultar complicada la toma de decisiones durante el proceso de tratamiento del paciente. Además, tratar eficazmente el problema y adaptarse a las circunstancias personales de cada uno de ellos, puede suponer alargar la vida o incluso salvar a la persona. Por ello, realizar una gestión y organización de los datos eficaz, y contar con herramientas que permitan un fácil acceso y visualización, pueden resultar fundamentales en la toma de decisiones de los tratamientos.


En la actualidad el \textbf{HUBU} almacena los datos de los pacientes que han sufrido o sufren cáncer de pulmón en formato texto. Este sistema de visualización conlleva a un análisis poco eficaz de los datos, ya que supone un proceso lento y costoso.

Para solventar este problema, el objetivo de este proyecto ha sido crear una \textbf{aplicación web} que sirva como \textbf{interfaz} para gestionar todos los datos de los pacientes. Estos estarán almacenados dentro de una \textbf{base de datos relacional} y la web tendrá herramientas de \textbf{visualización} de los datos como pueden ser gráficas o tablas permitiendo así, un óptimo análisis de dichos datos y facilitando la toma de decisiones durante los procesos de tratamiento.

También debido a que es posible que los médicos que trabajen con esta  \textbf{aplicación web} no tengan un alto nivel de informática se ha intentado realizar de la manera más \textbf{intuitiva} posible, con muchos  \textbf{controles de errores} y con una comunicación constante con el equipo de oncólogos del Hospital Universitario de Burgos para lograr la mejor versión posible.

\section{Material ajunto}

En el proyecto como material adjunto se incluye: 
\begin{itemize}
	\item Memoria
	\item Anexos 
	\item Aplicación web en \textit{Laravel}.
	\item Scripts \textit{SQL} para generar la base de datos e insertar datos necesarios para la aplicación.
	\item Script en \textit{Python} para generar la base de datos sintética.
\end{itemize}

Todos los elementos desarrollados durante el proyecto se pueden ver en el repositorio de \textit{GitHub}: \url{https://github.com/CarlosOrtu/TFG_HUBU}

