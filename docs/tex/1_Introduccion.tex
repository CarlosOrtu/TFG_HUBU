\capitulo{1}{Introducción}
El \textbf{cáncer} es un conjunto de enfermedades relacionadas las cuales son provocadas por una anomalía en las células que ocasiona que estas empiecen a dividirse continuamente y provoquen un mal funcionamiento de la zona incluso llegándose a expandir a otras partes del cuerpo, a esto se le llama \textbf{metástasis}\cite{cancerweb:online}. El \textbf{cáncer de pulmón} ocurre cuando este problema surge en el tejido pulmonar, este cáncer es el que mas muertes provoca actualmente en el mundo tanto en hombres como en mujeres, siendo el tabaquismo la principal causa de este problema llegando a suponer hasta el 80\% de los casos\cite{cancerpulmon:online}.

Hay mucha variedad de \textbf{canceres de pulmón}, mucha variedad de \textbf{tratamientos} y mucha variedad de \textbf{herramientas de detección}, por lo tanto tomar la mejor decisión puede llegar a ser complicado, aunque saber adaptarse a las circunstancias para poder tratar eficazmente este problema puede llegar a suponer alargar la vida del paciente o incluso llegar a salvarlo\cite{cancerpulmon2:online}. Para esto tener los datos de casos anteriores bien organizados y de fácil acceso y visualización puede ser un factor fundamental para tomar esta decisión de la manera más óptima posible.

En la actualidad el \textbf{Hospital Universitario de Burgos} almacena los datos de los pacientes que han sufrido o sufren cáncer de pulmón en formato texto, lo cual hace que la visualización de estos datos sea muy lenta y costosa y por lo tanto esto conlleva a un lento y poco eficiente análisis de los datos.

Para solventar este problema el objetivo de este proyecto ha sido crear una \textbf{aplicación web} que sirva como \textbf{interfaz} para gestionar todos los datos de los pacientes, los cuales van a estar almacenados dentro de una \textbf{base de datos relacional} y añadir herramientas de \textbf{visualización} de los datos como pueden ser gráficas o tablas que permitan un mejor análisis para facilitar la toma de decisiones.

También debido a que es posible que los médicos que trabajen con esta  \textbf{aplicación web} no tengan un alto nivel de informática se ha intentado realizar de la manera más \textbf{intuitiva} posible, con muchos  \textbf{controles de errores} y con una comunicación constante con el equipo de oncólogos del Hospital Universitario de Burgos para lograr la mejor versión posible.

\section{Material ajunto}

En el proyecto como material adjunto se incluye: 
\begin{itemize}
	\item Memoria
	\item Anexos 
	\item Aplicación web en \textit{Laravel}.
	\item Scripts \textit{SQL} para generar la base de datos e insertar datos necesarios para la aplicación.
	\item Script en \textit{Python} para generar la base de datos sintética.
\end{itemize}

Todos los elementos desarrollados durante el proyecto se pueden ver en el repositorio de \textit{GitHub}: \url{https://github.com/CarlosOrtu/TFG_HUBU}

